\section{The Integers and the Real Numbers}

\begin{enumerate}
  \item
  \begin{align*}
  (x+y) + z &= x + (y + z)\\
  (x\cdot y) \cdot z &= x \cdot (y \cdot z)
  \end{align*}

  \item \begin{align*}
    x + y &= y + x\\
    x \cdot y &= y \cdot x
  \end{align*}

  \item $\exists$ element of $\mathbb{R}$ called \textbf{zero}, denoted by $0$, such that
  \begin{equation*}
    x + 0 = x, \forall x \in \mathbb{R}
  \end{equation*}

  \item $\exists$ element of $\mathbb{R}$ called \textbf{one}, denoted by $1$, such that
  \begin{equation*}
    x \cdot 1 = x, \forall x \in \mathbb{R}
  \end{equation*}

  \item \begin{equation*}
    x \cdot (y + z) = (x \cdot y) + (x \cdot z)
  \end{equation*}

  \item \begin{align*}
    &\text{If } x>y, \text{ then } x + z > y + z\\
    &\text{If } x>y \text{ and } z > 0, \text{ then } x \cdot z > y \cdot z
  \end{align*}
\end{enumerate}

\begin{itemize}
  \item Properties 1-5 define a field
  \item Properties 1-6 define an ordered field
  \item Properties 7-8 by themselves define a linear continuum
\end{itemize}

\begin{definition}
  A \textbf{section} $S_n$ is defined as
  \begin{equation}
    S_{n+1} = \pbrac{
      1, 2, \dots, n
    }
  \end{equation}
\end{definition}

\section*{Exercises}

\bx{
  \TODO: Too lazy
}

\bx{
  \TODO: too lazy
}

\bx{
  \ea{
    \item If we take the intersection of any two inductive sets, call this
    intersection $A$, we know that $1 \in A$, and we still can inductively show
    that for any $x \in A$, that $x + 1 \in A$ as well, since $x$ is part of two
    inductive sets.
    We can then use induction to prove that the intersection of an arbitrary
    collection of inductive sets is also inductive.

    \item Proving the basic properties \begin{enumerate}[label=(\arabic*)]
      \item $1 > 0$ and $1 \in \mathbb{Z}$, so $1 \in \Zp$. Now for any
      $x \in \Zp$, the number $x+1$ is still an integer, and since $x >
      0$, we know $x + 1 > 0$, which means $x + 1 \in \Zp$ as well.
      Therefore, we conclude $\Zp$ is inductive.

      \item \AFSOC $A \neq \Zp$. Then we know that $\exists x \in
      \Zp, x \not\in A$. However, if we use induction from 1, we can
      show that $x \in A$, which is a contradiction. Therefore we conclude $A =
      \Zp$.
    \end{enumerate}
  }
}

\bx{
  \ea{
    \item We have $a \in \Zp$. We just want to inductively show that $a + x \in \Zp, \forall x \in \Zp$.
    We can do this by induction.
    \label{chap1:sec4:prob5:parta}
    \item Induction on $b$ \label{chap1:sec4:prob5:partb}
    \item We can either casework here, with $a = 0, a \geq 1$, or we can show that $a-1$ is inductive.
    \label{chap1:sec4:prob5:partc}
    \item We can use induction on $d$, via \ref{chap1:sec4:prob5:partc}
    \item We can use induction on $d$, both ways (positive and negative directions)
  }
}

\bx{
  \begin{enumerate}
    \item $a^na^m = a^{n+m}$, shown by \ref{chap1:sec4:prob5:parta}
    \label{chap1:sec4:prob6:parta}
    \item $(a^n)^m = a^{nm}$, shown by \ref{chap1:sec4:prob5:partb}
    \item $a^mb^m = (ab)^{m}$, induction on $m$, and you have to use the
    property in \ref{chap1:sec4:prob6:parta} for $(ab)^m \cdot ab = (ab)^{m+1}$.
  \end{enumerate}
}

\bx{
  \TODO: probably just doing induction in another direction
}

\bx{
  \ea{
    \item This is a corollary of the Axiom of Completness, which states that every nonempty set of real numbers that is bounded above has a least upper bound.
    \item Informally, suppose we have some $\epsilon > 0$ that is the $\inf$ of this set, then we can always find $1/n < \epsilon$, so therefore this $\epsilon$ is not the $\inf$, and we conclude that $0$ is the $\inf$.
    \item \TODO: induction proof, idea is that $a^n$ gets smaller as $n$ increases, and can be smaller than any $\epsilon$. We have to show this property by induction.
  }
}

\bx{
  \ea{
    \item Follows from that $\mathbb{Z} \subset \mathbb{R}$.
    \item \AFSOC there are two such numbers, $n_1, n_2 \in \mathbb{Z}$. WLOG $n_1 > n_2$, Then we can find that
    \begin{align*}
      n_1 &< x < n_1 + 1\\
      n_2 < x < n_2 + 1 \implies
      -n_2 - 1 &< -x < -n_2\\
      \implies n_1 - n_2 - 1 &< 0 < n_1 - n_2 + 1
    \end{align*}
    which is impossible, since $n_1 - n_2 - 1 \geq 0$, which is $\not< 0$, so
    therefore we cannot have more than one $n$.

    Now to show the existence of this $n$, we can use a contradiction argument
    along with an induction argument to show no such $n$ exists would imply $x
    \not\in \mathbb{R}$.

    I'm sure there's a more straightforward proof with lower bounds...using the
    fact that $\mathbb{Z}$ has a greatest lower bound.

    \item We have $n \geq x > y + 1 > y \geq m$ for $m, n \in \mathbb{Z}$.

    Either $y+1$ is an integer, in which case $n = y+1$. If $y+1$ is not an
    integer, then we know it is bounded below by some $b \in \mathbb{Z}$ such that $b < y + 1 < b + 1$,
    which we know must be $<x$, and $>y$, since if it were $\leq y$, then we could choose
    $b' = b + 1$, and $b' > y$ and $b' \leq y+1$, and thus $b$ would not satisfy $b + 1 > y+1$.
    In this case, $n = b$, and we have $x > n > y, n \in \mathbb{Z}$.

    \label{chap1:sec4:prob9:partc}

    \item We know that $x - y > 0$, so we can find some $n \in \mathbb{Z}$ such that $n(x - y) > 1$.
    Using \ref{chap1:sec4:prob9:partc}, we can find some $m \in \mathbb{Z}$ such that
    \begin{align*}
      nx &> m > ny\\
      x &> \frac{m}{n} > y
    \end{align*}

    let $z = m/n$.
  }
}

\bx{
    \ea{
      \item First, let's show
      \begin{align*}
        (x+h)^2 = x^2 + 2xh + h^2 &\leq x^2 + h(2x + 1)\\
        \implies h^2 &\leq h\\
        \implies h &\leq 1\tag{Given, since $h<1$}
      \end{align*}

      Notice in the last step, if $h = 0$, then we don't divide and $0^2 \leq 0$ holds.

      \begin{align*}
        (x-h)^2 = x^2 - 2xh + h^2 &\geq x^2 - 2xh\\
        \implies h^2 &\geq 0\\
        \implies h &\geq 0\tag{Given, since $h\geq 0$}
      \end{align*}

      Notice in the last step, if $h = 0$, then we don't divide and $0^2 \geq 0$ holds.

      \item If we have $x^2 < a$, then we know $(x+h)^2 \leq x^2 + h(2x + 1)$, so we can find an $h$ by considering the RHS and making it $< a$.
      We can do this by solving for $h$, which gives us
      \begin{equation*}
        h < \frac{a - x^2}{2x+1}
      \end{equation*}

      The other part of the problem is more of the same algebra.

      \item $B$ is bounded above by $1 + a$, and $a/2 \in B$.
      If we let $b = \sup B$, then $b^2$ cannot be $<a$, or else we can find
      some $(b + h)^2 < a, h > 0$, and show that this $b$ is not the $\sup$.

      \item \TODO: I'm stuck
    }
}

\bx{
  \ea{
    \item We know from exercises before that since $m/2 \not\in \mathbb{Z}$
    since $m$ is odd, we know $\exists n \in \mathbb{Z}$ such that $n < m/2 < n+1$.
    Now consider,
    \begin{equation*}
      2n < m < 2n+2
    \end{equation*}

    Since $m \in \mathbb{Z}$, we conclude $m = 2n+1$.

    \item $p \cdot q = (2m+1) \cdot (2n+1) = 4mn + 2m + 2n + 1 = 2(2mn + m + n) + 1$, which is in the form $2N+1$, so therefore $p \cdot q$ is odd.
    For $p^n$ to be odd, we show by induction.

    \item Honestly I think this is by definition of $\mathbb{Q}$. The hint in the textbook implies the proof should be something about finding the denominator in reduced form.
    \item I really hope you can show $\sqrt{2}$ is irrational by the time you take topology...
  }
}