\section{Fundamental Concepts}

\bx{
  We will check $\cup, \cap$ in DeMorgan's laws.

  Let's use
  \begin{itemize}
    \item $A = \pbrac{1, 2, 3, 4}$
    \item $B = \pbrac{-1, 2, 3, 5}$
    \item $C = \pbrac{3, 9, 11}$
  \end{itemize}

  Check
  \begin{align*}
    A - (B \cup C) &= \pbrac{1, 2, 3, 4}  - \pbrac{-1, 2, 3, 5, 9, 11}\\
    &= \pbrac{1, 4} \\
    &= (A - B) \cap (A - C) \\
    &= \pbrac{1, 4} \cap \pbrac{1, 2, 4} = \pbrac{1, 4}
  \end{align*}

  \begin{align*}
    A - (B \cap C) &= \pbrac{1, 2, 3, 4}  - \pbrac{3}\\
    &= \pbrac{1, 2, 4} \\
    &= (A - B) \cup (A - C) \\
    &= \pbrac{1, 4} \cup \pbrac{1, 2, 4} = \pbrac{1, 2, 4}
  \end{align*}
}

\bx{
  \ea{
    \item $\implies$ is true. $\impliedby$ is not true, consider $A = \pbrac{1, 2, 3}, B = \pbrac{1, 3}, C = \pbrac{2}$.
    \item $\implies$ is true. $\impliedby$ is not true, consider $A = \pbrac{1, 2, 3}, B = \pbrac{1, 3}, C = \pbrac{2}$.
    \item True.
    \item $\implies$ is not true. Consider $A = \pbrac{1} \subset B = \pbrac{1, 2}, C = \emptyset$. $\impliedby$ is true.
    \item Not true. Consider $A = \pbrac{1}, B = \pbrac{2}$. I think $\subset$ works.
    \item Not true. Consider $A = \pbrac{1, 2}, B = \pbrac{2, 3}$. LHS is equivalent to $A$, so this should be $\supset$.
    \item True.
    \item $\supset$
    \item True.
    \item True.
    \item Not true, if $A = \emptyset$ for example, we have $(A \times B)
    \subset (C \times D) = \emptyset \subset (C \times D)$, but we can set $B$
    to whatever and this statement is still true, so we can make $B$ have an
    element that is not in $D$, and therefore $B \not\subset D$.
    \item True.
    \item $\subset$
    \item $\subset$
    \item True.
    \item I think this is true at first glance...at least $\subset$ looks good.
    \item $\supset$
  }
}

\bx{
  \ea{
    \item
    \begin{itemize}
      \item \textbf{Original}: If $x < 0$ then $x^2 - x > 0$. True.
      \item \textbf{Contrapositive}: If $x^2 - x \leq 0$ then $x \geq 0$. True.
      \item \textbf{Converse}: If $x^2 - x > 0$ then $x < 0$. False.
    \end{itemize}

    \begin{figure}[H]
      \centering
      \def\domainSize{3}
      \begin{tikzpicture}
        \begin{axis}[
          axis y line = middle,
          axis x line = middle,
        ]

        \addplot[
          domain=-\domainSize:\domainSize,
          samples=100
        ]{
          x^2 - x
        };

        \end{axis}
      \end{tikzpicture}
      \caption{Showing how to visualize where $x^2 - x > 0 $}
      \label{chap1:fig:parabola}
    \end{figure}

    \item
    \begin{itemize}
      \item \textbf{Original}: If $x > 0$ then $x^2 - x > 0$. False.
      \item \textbf{Contrapositive}: If $x^2 - x \leq 0$ then $x \leq 0$. False.
      \item \textbf{Converse}: If $x^2 - x > 0$ then $x > 0$. False.
    \end{itemize}
  }
}

\bx{
  \ea{
    \item $\exists a \in A$ such that $a^2 \not\in B$
    \item $\forall a \in A, a^2 \not\in B$
    \item $\exists a \in A$ such that $a^2 \in B$.
    \item $\exists a \not\in A$ such that $a^2 \not\in B$.
  }
}

\bx{
  \ea{
    \item True. True.
    \item False. True.
    \item True. False.
    \item True. True.
  }
}

\bx{
  \TODO too lazy
}

\bx{
  \begin{align*}
    D &= A \cap (B \cup C) \\
    E &= (A \cap B) \cup C \\
    F &= A
  \end{align*}

  For $F$, I was thinking $x \in B \implies x \in C$ means that either $x \in B$
  and $x \in C$, or $x \not\in B$ and $x$ can be anything. This sounds like $x$
  can be anything in the second case, so we have $A \cap \mathcal{U} = A$.
}

\bx{
  $A = \pbrac{0, 1}$. $\mathcal{P}(A) = \pbrac{\emptyset, \pbrac{0}, \pbrac{1}, \pbrac{0, 1}}$.

  If $A$ has one element, $\abs{\mathcal{P}(A)} = 2$. It is called the power set
  because it contains all the subsets of $A$, and that $\abs{\mathcal{P}(A)} =
  2^\abs{A}$.
}

\bx{
  \TODO: You can honestly find this everywhere online. Standard proof.
}

\bx{
  \ea{
    \item $\mathbb{Z} \times \mathbb{R}$
    \item $\mathbb{R} \times \left(0, 1\right]$
    \item No. You can do a contradiction proof with cases that the first and second set are disjoint, and then that they are not disjoint.
    \item Yes, $\pa{\mathbb{R} - \mathbb{Z}} \times \mathbb{Z}$
    \item No. The cartesian product will produce a box, while this set is a circle.
  }
}