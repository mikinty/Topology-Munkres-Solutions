\section{Countable and Uncountable Sets}

\begin{definition}
  A set $A$ is said to be \textbf{infinite} if it is not finite. It is said to
  be \textbf{countably infinite} if there is a bijective correspondence
  \begin{equation}
    f:A \to \Zp
  \end{equation}
\end{definition}

\begin{definition}
  A set is said to be \textbf{countable} if it is either finite or countably
  infinite. A set that is not countable is said to be \textbf{uncountable}.
\end{definition}

\begin{theorem}
  Let $B$ be a nonempty set, then the following are equivalent:
  \begin{enumerate}[label=(\arabic*)]
    \item $B$ is countable
    \item There is a surjective function $f : \Zp \to B$
    \item There is an injective function $g : B \to \Zp$
  \end{enumerate}
\end{theorem}

\begin{theorem}
  A countable union of countable sets is countable.
\end{theorem}

\begin{theorem}
  A finite product of countable sets is countable.
\end{theorem}

\section*{Exercises}

\bx{
  Lazy way to prove this, but you can make an injective function
  $f : \mathbb{Q} \to \Zp \times \Zp \times \Zp$
  where given any
  \begin{equation*}
    f(\pm \frac{m}{n}) = (m, n, \pbrac{1 \text{ if negative}, 2 \text{ if positive}})
  \end{equation*}

  And then we know that $\Zp \times \Zp \times \Zp$
  is a finite product of countable sets, so it is also countable. Since $f$ is
  an injective function into a countably infinite set, we conclude that
  $\mathbb{Q}$ is also countably infinite.
}

\bx{
  Checking for bijections. I'm not going to explicitly show this, but you can find inverse functions to show bijectivity.
  \TODO.
}

\bx{
  The bijection is whether or not some $n \in \Zp$ is included in a set
  or not, which corresponds to 0 (not included) and 1 (included) in the tuple of
  $X^\omega$.
}

\bx{
  \ea{
    \item For a given $n$, the number of algebraic numbers is $\prod_{i=0}^{n-1}
    \mathbb{Q}$. Then taking the union over $\Zp$, this would be a
    countable union of countable sets, which is also countable. Therefore there
    are a countable number of algebraic numbers.
    \item AFSOC transcendental numbers are countable. Then the real numbers are
    a union of two countable sets, which could make $\mathbb{R}$ countable,
    which is a contradiction.
  }
}

\bx{
  \ea{
    \item This set of functions can be bijected to $\Zp \times \Zp$, which is countable.
    \item This would be $\prod_{i=1}^n \Zp$, which is a finite product of countable sets, which is countable.
    \item This is a countable union of countable sets, which is countable.
    \item Not countable, you can use diagonalization to construct a function that is not included in a countable map.
    \item Not countable, can also use a diagonalization argument to construct a function that differs in the $i^\text{th}$ spot for $f_i$.
    \item Countable, since this is essentially a subset of the countable union of $\Zp$. \label{chap1:sec7:prob5:partf}
    \item Countable, same reasoning as \ref{chap1:sec7:prob5:partf}
    \item This is a countable union of a countable union of countable sets, which is still countable.
    \item Countable, this is a subset of $\Zp \times \Zp$.
    \item Countable, this is just $\cup_{i=0}^\infty \prod_{j=1}^i \Zp$, which is a countable union of a finite product of countable sets.
  }
}

\bx{
  \ea{
    \item Originally I was going to define the identity map $I : B \to A$, since
    $B \subset A$, and say that injections both ways implies bijection, but
    that's the theorem we are trying to prove in the next step, so let's just do
    what the problem tells us here.

    To understand the hint a bit more, the reason we can plug in $B$ values for
    $f$ is because $B \subset A$. And the reason that we get an $A_i$ value from
    the result of $f$ is also because $B \subset A$.

    Now, if we define this new function $h$,
    \begin{equation}
      h(x) = \begin{cases}
        f(x) &\text{if $x \in A_n - B_n$}\\
        x &\text{otherwise}
      \end{cases}
    \end{equation}

    We have to check this function is bijective.
    \begin{itemize}
      \item \textbf{Injective}: We have 2 cases for $x$. Either it is in $B$, in which case we just spit out $x$, or it is in $A_n$ but not in $B_n$, in which case we apply $f(x)$, and make $x \in B$.
      This is injective, because cases, for some $x_1, x_2 \in A$,
      \begin{enumerate}
        \item $x_1, x_2 \in A_n - B_n$, then we are just looking at $f$, which we know is injective
        \item $x_1 \in A_n - B_n, x_2 \not\in A_n - B_n$, then $x_1 \neq x_2$,
        and since $x_1 \in A_n - B_n$, it is never mapped to itself, because it
        if were, then it would imply it is $\not\in A_n - B_n$, which is a
        contradiction.
        \item $x_1, x_2 \not\in A_n - B_n$, then we are using the identity map, which is injective
      \end{enumerate}

      \item \textbf{Surjective}: This is the more difficult part of the proof, but also the more fun.

      The idea of the recursive definition is that we want to partition $B$ into parts, $A_i, B_i$.
      The only part that we have trouble with is $A_1 - B_1$, since this part is
      not in $B$. However, by applying $f$ to this portion, we can have it map
      into $B$, which helps us map $B$. However, since $f(A_1)$ ``takes the
      spot'' of $A_2$, we then apply $f$ on $A_2$, so that $f(A_2) \subset B$.
      $f(A_2)$ takes the spot of $A_3$, so we keep applying this recursive
      definition. Eventually, we can see that all of $B$ is mapped, so $h$ is
      also surjective on $B$.
      I know this is an informal proof, but Figure
      \ref{chap1:sec7:prob6:parta:fig} explains the hardest part of the proof.

      \begin{figure}[H]
        \centering
        \begin{tikzpicture}
          \node[rectangle, draw, fill=gray!40] (r_1) at (0, 0) {$A_1$};
          \node[rectangle, draw, anchor=north west] (r_2) at (r_1.north east) {$B_1$};
          \node[rectangle, draw, anchor=north west, fill=gray!40] (r_3) at (r_2.north east) {$A_2$};
          \node[rectangle, draw, anchor=north west] (r_4) at (r_3.north east) {$B_2$};
          \node[rectangle, draw, anchor=north west, fill=gray!40] (r_5) at (r_4.north east) {$A_3$};
          \node[rectangle, draw, anchor=north west] (r_6) at (r_5.north east) {$B_3$};
          \node[rectangle, draw, anchor=north west, fill=gray!40] (r_7) at (r_6.north east) {$A_4$};
          \node[rectangle, draw, anchor=north west] (r_8) at (r_7.north east) {$B_4$};

          \draw[-stealth] (r_1.north) to[bend left=60] node[above] {$f$} (r_3.north);
          \draw[-stealth] (r_3.north) to[bend left=60] node[above] {$f$} (r_5.north);
          \draw[-stealth] (r_5.north) to[bend left=60] node[above] {$f$} (r_7.north);

          \draw[-stealth] (r_2.south) -- ++(0, -0.5);
          \draw[-stealth] (r_3.south) -- ++(0, -0.5);
          \draw[-stealth] (r_4.south) -- ++(0, -0.5);
          \draw[-stealth] (r_5.south) -- ++(0, -0.5);
          \draw[-stealth] (r_6.south) -- ++(0, -0.5);
          \draw[-stealth] (r_7.south) -- ++(0, -0.5);
          \draw[-stealth] (r_8.south) -- ++(0, -0.5);

          \draw ($(r_2.south west) + (0, -0.5)$) rectangle
            ($(r_8.south east) + (0, -1)$) node[pos=.5] {$B$};
        \end{tikzpicture}
        \caption{Showing the subset partitions for $A_i, B_i$}
        \label{chap1:sec7:prob6:parta:fig}
      \end{figure}
    \end{itemize}

    \textbf{Note:} To be honest, I'm not sure why we had to go through the
    trouble of defining the recursive definition. To me, it seems like with $f$,
    the scenario is that $A$ maps to $B$ injectively, except there are some some
    elements in $B$ that are not mapped.

    \textbf{Update:} It seems like the reason is that with a naive definition of
    just $h(x)$ mapping $f(x)$ for $x \in A-B$ and identity for $x \in A \cap
    B$, we cannot claim that this combo of $f(x), x$ maps $B$ surjectively. See
    the surjective part of the proof to see why this proof is so fun.

    \label{chap1:sec7:prob6:parta}

    \item We have two cases, either $A \subset C$ or $C \subset A$. In either
    case, we can apply part \ref{chap1:sec7:prob6:parta}.
  }
}

\bx{
  We can see that $E \subset D$. Need $f: D \to E$. We can construct such an $f$
  by representing each $x \in \Zp$ in binary, and mapping it to the
  corresponding 0-1 tuple in $\pbrac{0, 1}$.
}

\bx{
  \TODO. I'm stuck. I assume we want to do the double bijective proof to show equal cardinalities.
}

\bx{
  \ea{
    \item First rearrange, the formula to be easier to apply:
    \begin{equation*}
      h(n+1) = \sqrt{
        h(n) + h(n-1)^2
      }
    \end{equation*}
    Trying out some values,
    \begin{align*}
      h(3) &= \sqrt{2 + 1} = \sqrt{3}\\
      h(4) &= \sqrt{4 + \sqrt{3}}\\
      h(5) &= \sqrt{3 + \sqrt{4 + \sqrt{3}}}
    \end{align*}
    so we know that there exists such a function

    \item Notice that when we took the square root, we could have had $\pm$, so there it is not well defined which one we should take.
    \item If we try to solve for
    \begin{align*}
      h(3) &= \sqrt{
        2 - 1
      } = \pm 1\\
      h(4) &= \sqrt{
        \pm 1 - 2
      } \implies \text{square root of negative number...}
    \end{align*}
    imaginary numbers? Never heard of them.
  }
}