\section{Functions}

\bx{
  \ea{
    \item Suppose $x \in A_0$. Then consider

    \begin{equation*}
      f^{-1}(f(A_0)) = \pbrac{x' \mid f(x') \in f(A_0)}.
    \end{equation*}

    Since $x \in A_0$, we know $f(x) \in f(A_0)$, and so we can conclude $x \in f^{-1}(f(A_0))$.

    If $f$ is injective, then we know $f(x) = f(x'), x' \in A_0$ implies that $x = x' \implies x \in A_0$, proving equality.

    \item Suppose $y \in f(f^{-1}(B_0)) = \pbrac{y' \mid \exists x \in f^{-1}(B_0), y' = f(x)}$.

    We have $x \in \pbrac{x' \mid f(x') \in B_0}$, which means $y' = f(x) \in B_0$, so therefore $y \in B_0$.

    If $f$ is surjective, then we know $y \in B_0 \implies \exists x \in A$ such that $f(x) = y$.

    In particular, $f(x) = y \in B_0$, this set of $x \in f^{-1}(B_0)$, so we can conclude that $y \in f(f^{-1}(B_0))$.
  }
}

\bx{
  \ea{
    \item We can write some definitions first
    \begin{itemize}
      \item $f^{-1}(B_0) = \pbrac{x \mid f(x) \in B_0}$
      \item $f^{-1}(B_1) = \pbrac{x \mid f(x) \in B_1}$
    \end{itemize}

    If we know $B_0 \subset B_1$, then for some $x \in B_0$, we know $x \in B_1$.

    This means for some $f(x) \in B_0, f(x) \in B_1$ as well, so therefore $f^{-1}(B_0) \subset f^{-1}(B_1)$.

    \item
    \begin{align*}
      f^{-1}(B_0 \cup B_1) &= \pbrac{x \mid f(x) \in B_0 \cup B_1}\\
      &= \pbrac{x \mid f(x) \in B_0 \text{ or } f(x) \in B_1}\\
      &= \pbrac{x \mid f(x) \in B_0} \cup \pbrac{x \mid f(x) \in B_1}\\
      &= f^{-1}(B_0) \cup f^{-1}(B_1)
    \end{align*}

    \label{item:inverse_or}

    \item Basically the same proof as \ref{item:inverse_or}.
    \item Basically the same proof as \ref{item:inverse_or}.
    \item Suppose $x \in A_0$ means $x \in A_1$ as well.
    Consider $y \in f(A_0) = \pbrac{y' \mid y' = f(x) \text{ for some } x \in A_0}$. Because of our assumptions, it is also the case that
    $y \in \pbrac{y' \mid y' \text{ for some } x \in A_1} = f(A_1)$.

    Notice that $f(A_0) \subset f(A_1)$ does not imply that $A_0 \subset A_1$. E.g. think parabola.

    \item
    \begin{align*}
      f(A_0 \cup A_1)
      &= \pbrac{y \mid y = f(x) \text{ for some } x \in A_0 \cup A_1} \\
      &= \pbrac{y \mid y = f(x) \text{ for some } x \in A_0 \text{ or } x \in A_1} \\
      &= \pbrac{y \mid y = f(x) \text{ for some } x \in A_0} \cup \pbrac{y \mid y \text{ for some } x \in A_1} \\
      &= f(A_0) \cup f(A_1)
    \end{align*}

    \item
    \begin{equation*}
      f(A_0 \cap A_1) = \pbrac{y \mid y = f(x) \text{ for some } x \in A_0 \cap A_1}
    \end{equation*}
    which implies that $f(A_0 \cap A_1) \in f(A_0)$, since $x \in A_0$ and $f(A_0 \cap A_1) \in f(A_1)$, since $x \in A_1$.

    This means that $f(A_0 \cap A_1) \subset f(A_0) \cap f(A_1)$.

    Now, if $f$ is injective, then if we start with $y \in f(A_0) \cap f(A_1)$,
    we know $y \in \pbrac{y' \mid y' = f(x), x \in A_0}$ and
    $y \in \pbrac{y' \mid y' = f(x), x \in A_1}$. Since $f$ is injective, the
    common $y'$ values in $f(A_0)$ and $f(A_1)$ will map to the same $x$ values
    in $A_0$ and $A_1$, which means $y \in \pbrac{y' \mid y' = f(x), x \in A_0 \cap A_1} = f(A_0 \cap A_1)$.
  }
}

\bx{
  too lazy
}

\bx{
  \ea{
    \item Consider
    \begin{align*}
      (g \circ f)^{-1}(C_0)
      &= \pbrac{a \mid (g \circ f)(x) \in C_0} \tag{by definition}\\
      &= \pbrac{a \mid \text{for some } b \in B, f(a) = b, g(b) = c \in C_0} \\
      &= \pbrac{a \mid \text{for some } b \in B, f(a) = b, b \in g^{-1}(C_0)} \tag{we know $b \in g^{-1}(C_0)$ since $C_0 \subset C$}\\
      &= \pbrac{a \mid f(a) \in g^{-1}(C_0)}\\
      &= f^{-1}\pa{g^{-1}\pa{C_0}}
    \end{align*}

    Just a note for this problem, it's easy to get caught up with definitions
    and forget why we need assumptions. It might seem easy to do this problem
    without the fact that $C_0 \subset C$, but if you look at the step where we
    use that property, if $C_0 \not\subset C$, we cannot assume that $b \in
    g^{-1}(C_0)$, e.g. if $C_0$ contains elements that are not in $C$.

    \item Suppose we have
    \begin{align*}
      (g \circ f)(a) &= (g \circ f)(a')\\
      g(f(a)) &= g(f(a'))\tag{by def.}\\
      f(a) &= f(a')\tag{because $g$ is injective}\\
      a &= a'\tag{because $f$ is injective}
    \end{align*}
    therefore we conclude that $g \circ f$ is also injective.

    \item If we know that $g \circ f$ is injective,
    \begin{itemize}
      \item \AFSOC $f$ is not injective. Then $\exists a_1, a_2$ such that
      $f(a_1) = f(a_2)$ but $a_1 \neq a_2$. If this is the case, then $(g \circ
      f)(a_1) = g(f(a_1)) = g(f(a_2)) = (g \circ f)(a_2)$ which shows that $g
      \circ f$ is not injective. Which is a contradiction. Therefore, $f$ must
      be injective.

      \item It is possible for $g$ to not be injective. We can have some $b \in
      B$ that $\not\exists a \in A$ such that $f(a) = b$. In this case, we will
      not be able to find some input $a \neq a' \in $ where we break injectivity
      for $g \circ f$.
    \end{itemize}

    \item Suppose $f$ and $g$ are surjective. Now consider some $c \in C$.
    Since $g$ is surjective, we know $\exists b \in B$ such that $g(b) = c$.
    For this $b \in B$, since $f$ is surjective, we know that $\exists a \in A$
    such that $f(a) = b$.
    This means for any $c \in C$, we know $\exists a$ such that $g(f(a)) = (g
    \circ f)(a) = c$, which means $g \circ f$ is surjective.

    \item If we know that $g \circ f$ is surjective,
    \begin{itemize}
      \item It is possible for $f$ to not be surjective. Intuitively, a
      counterexample would show that there is some $b \in B$ such that
      $\not\exists a \in A$ such that $f(a) = b$. But all we have to make sure
      in our example is that whatever $g(b) = c$ maps to, $\exists a' \neq a \in
      A$ such that $(g \circ f)(a') = c$.

      \item \AFSOC $g$ is not surjective. Then $\exists c \in C$ such that
      $\not\exists b \in B$ such that $g(b) = c$. If this is the case, then
      $\not\exists a \in A$ such that $(g \circ f)(a) = c$, which means $g \circ
      f$ is not surjective. This is a contradiction, therefore $g$ must be
      surjective.
    \end{itemize}

    \item Summary should be pretty clear :) from the results above.
  }
}

\bx{
  \item Alright let's consider the two cases in this problem
  \begin{itemize}
    \item $f$ has a left inverse, i.e. $\exists g, g \circ f = i_A$.

    \AFSOC $f$ is not injective. This means that $f(a) = f(a') = b \in B$ but $a \neq a' \in A$.

    If this is the case, then $g(b) = a$ or $g(b) = a'$, but cannot be both, by the definition of a function, which means that $g \circ f$ is not $i_A$ by counterexample of either $a$ or $a'$.

    \item $f$ has a right inverse, i.e. $\exists h, f \circ h = i_B$.

    \AFSOC $f$ is not surjective. Then $\exists b \in B$ such that $\not\exists a \in A$ such that $f(a) = b$. If this is the case, then for this $b$, $(f \circ h)(b) \neq b$, in which case $f \circ h \neq i_B$.
  \end{itemize}

  \label{chap1:p5:part:a}

  \item The proofs above in \ref{chap1:p5:part:a} are by contradiction, and illustrate how to construct such a counterexample.
  \item The proofs above in \ref{chap1:p5:part:a} are by contradiction, and illustrate how to construct such a counterexample.
  \item No, these left/right inverses are unique. Not rigorous, but if you have an identity mapping and you change any of the mappings, it will no longer be an identity mapping.
  \item So we have $f$ that has both a left and right inverse, $g, h$
  respectively.In that case, we know that $f$ is injective and surjective, by
  the results of \ref{chap1:p5:part:a}, so we can conclude that $f$ is
  bijective. We have $g = h = f^{-1}$ by Lemma 2.1 in the text.

  (We could've just used Lemma 2.1 directly, but I think it's important to
  remind ourselves that showing a function is injective and surjective is a
  problem solving technique for showing a function is bijective.)
}

\bx{
  Let us draw out our function first,

  \begin{figure}[H]
    \centering
    \def\domainSize{2}
    \begin{tikzpicture}
      \begin{axis}[
        axis y line = middle,
        axis x line = middle,
      ]

      \addplot[
        domain=-\domainSize:\domainSize,
        samples=100
      ]{
        x^3 - x
      };

      \end{axis}
    \end{tikzpicture}
    \caption{Plotting $f(x) = x^3 - x$}
    \label{chap1:sec2:p6:fig:1}
  \end{figure}

  A less formal way to imagine injective functions is to use the horizontal line
  test. If you sweep a horizontal line and it intersects the plot in more than 1
  spot, then you know the function is not injective. We see our function $f(x)$
  here has issues between $[-1, 1]$.

  For surjectivity, we just need to map all of $\mathbb{R}$ on the $y$-axis. Our
  function does this nicely already, so we just have to make our function
  injective, and then make sure to keep the surjectivity.

  We have two choices here, we can either restrict our domain to be
  \begin{itemize}
    \item $D = (-\infty, -1) \cup [-1, \infty)$
    \item $D = (-\infty, -1] \cup (-1, \infty)$
  \end{itemize}

  If we use the first choice, we will get the following plot as $g$:
  \begin{figure}[H]
    \centering
    \begin{tikzpicture}
      \begin{axis}[
        axis y line = middle,
        axis x line = middle,
      ]

      \addplot[
        domain=-2:-1,
        samples=100
      ]{
        x^3 - x
      };

      \addplot[
        domain=1:2,
        samples=100
      ]{
        x^3 - x
      };

      \addplot[only marks, mark=o] table[row sep=crcr] {
        x y\\
        -1 0\\
      };

      \addplot[only marks] table[row sep=crcr] {
        x 0\\
        1 0\\
      };
      \end{axis}
    \end{tikzpicture}
    \caption{Plotting $g(x)$, a bijective function. Open circles are exclusive, closed circles are inclusive.}
    \label{chap1:sec2:p6:fig:2}
  \end{figure}

  To find $g^{-1}$, a classic algebra way to do this is to solve the function in
  terms of $x$. Because of the horizontal line issues from before, we will
  encounter, some issues, but because of our domain restriction, things should
  be ok.

  Instead of solving for $x$, since we already have a plot, we can just do a
  reflection across $y=x$, and we will get the inverse function $g^{-1}$. The
  intuitive way to think about this is that we are essentially swapping all $(x,
  y)$ coordinates to become $(y, x)$.

  \begin{figure}[H]
    \centering
    \begin{tikzpicture}
      \begin{axis}[
        axis y line = middle,
        axis x line = middle,
      ]

      \addplot[domain=-2:-1] (x^3 - x, x);

      \addplot[domain=1:2] (x^3 - x, x);

      \addplot[only marks, mark=o] table[row sep=crcr] {
        x y\\
        0 -1\\
      };

      \addplot[only marks] table[row sep=crcr] {
        x y\\
        0 1\\
      };
      \end{axis}
    \end{tikzpicture}
    \caption{Plotting $g(x)$, a bijective function. Open circles are exclusive, closed circles are inclusive.}
    \label{chap1:sec2:p6:fig:3}
  \end{figure}
}