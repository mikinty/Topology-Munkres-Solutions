\section{Cartesian Products}

\begin{definition}
  Let $\mathcal{A}$ be a nonempty collection of sets. An \textbf{indexing
  function} for $\mathcal{A}$ is a surjective function $f$ from some set $J$,
  called the \textbf{index set}, to $\mathcal{A}$. The collection $\mathcal{A}$, together with the indexing function $f$ is called an \textbf{indexed family of sets}. Given $\alpha \in J$, we shall denote the set $f(\alpha)$ by the symbol $A_\alpha$. And we shall denote the indexed family itself by the symbol
  \begin{equation}
    \pbrac{
      A_\alpha
    }_{
      \alpha \in J
    },
  \end{equation}
  which is read as ``the family of all $A_\alpha$, as $\alpha$ ranges over $J$.''
\end{definition}

\section*{Exercises}

\bx{
  Consider $f : A \to B$
  \begin{align*}
    f(a, b) &= (b, a)\\
    f^{-1}(b, a) &= (a, b)
  \end{align*}
}

\bx{
  I'm just going to write out functions that you can check are bijective.
  \ea{
    \item $f(a_1, a_2, \dots, a_n) = ((a_1, \dots, a_{n-1}), a_n)$
    \item \begin{align*}
      f_{-1} (a_1, a_2, \dots) &= ((a_1, a_2), (a_3, a_4), \dots) = (b_1, b_2, \dots)\\
      f^{-1} (b_1, b_2, \dots) &= f^{-1}((a_1, a_2), (a_3, a_4), \dots)\\
        &= (a_1, a_2, \dots)
    \end{align*}
  }
}

\bx{
  \ea{
    \item For any $b \in B$, we have that every index element $\in A_i$, so
    therefore $b \in A$. Therefore $B \subset A$.
    \item The converse is if $B \subset A$, then $B_i \subset A_i$. This is true
    because AFSOC $\exists b \in B$ such that $\exists i$ such that $B_i \not\subset
    A_i$. Then $b \not \in A$, and therefore $B \not\subset A$. But this is a
    contradiction.
    \item AFSOC $\exists i$ such that $A_i$ is empty. Then, $A$ must be empty,
    because there is no element that can be in the $i^\text{th}$ tuple. If every
    $A_i$ is nonempty, then $A$ must also be nonempty.
    \item \begin{itemize}
      \item $A \cup B \subset \prod_i A_i \cup B_i$
      \item $A \cap B \supset \prod_i A_i \cap B_i$
    \end{itemize}
  }
}

\bx{
  \ea{
    \item Just map the first $m$ tuples to the first $m$, and the remainder just map to any element of $X$
    \item $f((x_1, \dots, x_m), (x'_1, \dots, x'_n)) = (x_1, \dots, x_m, x'_1, \dots, x'_n)$
    \item Just map the first $n$ elements, and for the rest just use any element of $X$
    \item Map first $n$, then map the rest. Other way, Map first $n$, then map the rest
    \item $f((x_1, x_2, \dots), (x'_1, x'_2, \dots)) = (x_1, x'_1, x_2, x'_2, \dots)$
    \label{chap1:sec5:prob4:part3}
    \item Just cycle through the $n$ coordinates for mapping, similar to
    \ref{chap1:sec5:prob4:part3}.
  }
}

\bx{
  \ea{
    \item Yes, just do $\prod_i \mathbb{Z}$
    \item Easy, just $\prod_i [i, \infty)$
    \item $\prod_{i=1}^{100} \mathbb{R} \times \prod_{i=101}^\infty \mathbb{Z}$
    \item Cannot be expressed.
  }
}