\section{Relations}

\begin{definition}
  A \textbf{relation} on a set $A$ is a subset $C$ of the cartesian product $A \times A$.

  For a relation $C$ on $A$, we use the notation $xCy$ to mean $(x, y) \in C$, or ``$x$ is in the relation $C$ to $y$.''
\end{definition}

\begin{definition}
  An \textbf{equivalence relation} on a set $A$ is a relation $C$ on $A$ having the following 3 properties: (we use $\sim$ to denote the equivalence relation)
  \begin{enumerate}
    \item Reflexivity: $xCx \quad \forall x \in A$ ($x \sim x$)
    \item Symmetry: If $xCy$ then $yCX$ ($x \sim y \implies y \sim x$)
    \item Transitivity: If $xCy$ and $yCz$ then $xCz$ ($x \sim y \land y \sim z \implies x \sim z$)
  \end{enumerate}
\end{definition}

\begin{definition}
  We call a subset of $E$ of $A$ the \textbf{equivalence class} determined by $x$ as the equation
  \begin{equation}
    E = \pbrac{
      y \mid y \sim x
    }
  \end{equation}
\end{definition}

\begin{definition}
  A \textbf{partition} of a set $A$ is a collection of disjoint nonempty subsets of $A$ whose union is all of $A$.
\end{definition}

\begin{definition}
  A relation $C$ on a set $A$ is called an \textbf{order relation} if it has the following properties:
  \begin{enumerate}
    \item Comparability: For every $x, y \in A$ for which $x \neq y$, either $xCy$ or $yCx$
    \item Nonreflexitivity: For no $x \in A$ does $xCx$ hold
    \item Transtivity: If $xCy$ and $yCz$ then $xCz$
  \end{enumerate}
\end{definition}

\begin{definition}
  if $X$ is a set and $<$ is an order relation on $X$, and if $a<b$, we use the notation $(a, b)$ to denote the set
  \begin{equation}
    \pbrac{
      x \mid a < x < b
    };
  \end{equation}
  it is called an \textbf{open interval} in $X$. If this set if empty, we call
  \begin{itemize}
    \item $a$ the \textbf{immediate predecessor} of $b$
    \item $b$ the \textbf{immediate successor} of $a$
  \end{itemize}
\end{definition}

\begin{definition}
  Suppose $A, B$ are two sets with order relations $<_A$ and $<_B$ respectively. We say that $A, B$ have the same \textbf{order type} if there is a bijective correspondence between them that preserves order; that is, if there exists a bijective function $f : A \to B$ such that
  \begin{equation}
    a_1 <_A a_2 \implies f(a_1) <_B f(a_2)
  \end{equation}
\end{definition}

\begin{definition}
  An ordered set $A$ has the \textbf{least upper bound property} if every
  nonempty subset $A_0$ of $A$ that is bounded above has a least upper bound.
  Analogously, the set $A$ is said to have the \textbf{greatest lower bound
  property} if every nonempty subset $A_0$ of $A$ that is bounded below has a
  greatest lower bound.
\end{definition}

\section*{Exercises}

\bx{

}
