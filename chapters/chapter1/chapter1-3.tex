\section{Relations}

\begin{definition}
  A \textbf{relation} on a set $A$ is a subset $C$ of the cartesian product $A \times A$.

  For a relation $C$ on $A$, we use the notation $xCy$ to mean $(x, y) \in C$, or ``$x$ is in the relation $C$ to $y$.''
\end{definition}

\begin{definition}
  An \textbf{equivalence relation} on a set $A$ is a relation $C$ on $A$ having the following 3 properties: (we use $\sim$ to denote the equivalence relation)
  \begin{enumerate}
    \item Reflexivity: $xCx \quad \forall x \in A$ ($x \sim x$)
    \item Symmetry: If $xCy$ then $yCx$ ($x \sim y \implies y \sim x$)
    \item Transitivity: If $xCy$ and $yCz$ then $xCz$ ($x \sim y \land y \sim z \implies x \sim z$)
  \end{enumerate}
\end{definition}

\begin{definition}
  We call a subset of $E$ of $A$ the \textbf{equivalence class} determined by $x$ as the equation
  \begin{equation}
    E = \pbrac{
      y \mid y \sim x
    }
  \end{equation}
\end{definition}

\begin{definition}
  A \textbf{partition} of a set $A$ is a collection of disjoint nonempty subsets of $A$ whose union is all of $A$.
\end{definition}

\begin{definition}
  A relation $C$ on a set $A$ is called an \textbf{order relation} if it has the following properties:
  \begin{enumerate}
    \item Comparability: For every $x, y \in A$ for which $x \neq y$, either $xCy$ or $yCx$
    \item Nonreflexitivity: For no $x \in A$ does $xCx$ hold
    \item Transtivity: If $xCy$ and $yCz$ then $xCz$
  \end{enumerate}
\end{definition}

\begin{definition}
  if $X$ is a set and $<$ is an order relation on $X$, and if $a<b$, we use the notation $(a, b)$ to denote the set
  \begin{equation}
    \pbrac{
      x \mid a < x < b
    };
  \end{equation}
  it is called an \textbf{open interval} in $X$. If this set if empty, we call
  \begin{itemize}
    \item $a$ the \textbf{immediate predecessor} of $b$
    \item $b$ the \textbf{immediate successor} of $a$
  \end{itemize}
\end{definition}

\begin{definition}
  Suppose $A, B$ are two sets with order relations $<_A$ and $<_B$ respectively. We say that $A, B$ have the same \textbf{order type} if there is a bijective correspondence between them that preserves order; that is, if there exists a bijective function $f : A \to B$ such that
  \begin{equation}
    a_1 <_A a_2 \implies f(a_1) <_B f(a_2)
  \end{equation}
\end{definition}

\begin{definition}
  An ordered set $A$ has the \textbf{least upper bound property} if every
  nonempty subset $A_0$ of $A$ that is bounded above has a least upper bound.
  Analogously, the set $A$ is said to have the \textbf{greatest lower bound
  property} if every nonempty subset $A_0$ of $A$ that is bounded below has a
  greatest lower bound.
\end{definition}

\section*{Exercises}

\bx{
  Check equivalence relation:
  \begin{itemize}
    \item Reflexivity is obvious
    \item Equality is symmetric, so the relation is too
    \item Equality is transitive, so the relation is too
  \end{itemize}

  This looks like a bunch of parabolas of the form $y = x^2 + C$ on the plane, see figure \ref{chap1:sec3:p1:fig:1}

  \begin{figure}[H]
    \centering
    \def\domainSize{2}
    \begin{tikzpicture}
      \begin{axis}[
        axis y line = middle,
        axis x line = middle,
      ]

      \addplot[
        color=red,
        ultra thick,
        domain=-\domainSize:\domainSize,
        samples=100
      ]{
        x^2
      };

      \addplot[
        color=blue,
        ultra thick,
        domain=-\domainSize:\domainSize,
        samples=100
      ]{
        x^2 + 0.25
      };

      \addplot[
        color=green,
        ultra thick,
        domain=-\domainSize:\domainSize,
        samples=100
      ]{
        x^2 + 0.5
      };

      \addplot[
        color=black,
        ultra thick,
        domain=-\domainSize:\domainSize,
        samples=100
      ]{
        x^2 + 0.75
      };

      \end{axis}
    \end{tikzpicture}
    \caption{Plotting the partition defined by the equivalence relation}
    \label{chap1:sec3:p1:fig:1}
  \end{figure}
}

\bx{
  Reflexivity will still hold in $A_0$, since $A_0 \in A$, and $C$ applies
  to any element $x \in A$.
  Symmetry still holds, since $x, y \in A$, and transitivity also holds since
  $x, y, z \in A$. The idea is that $A_0$'s elements are contained in $A$, so
  all the equivalence relation properies still hold.
}

\bx{
  We are assuming $\exists b$ such that $aCb$. If there is no such $b$, then we do not have $aCa$.
}
\bx{
  \item \ea{
    \item Let us check the properties
    \begin{itemize}
      \item Reflexive: $f(a) = f(a)$ is trivial
      \item Symmetric: If we have $f(a) = f(b)$, then $f(b) = f(a)$
      \item Transitive: Equality is transitive, so this also holds
    \end{itemize}

    \item $A^\ast$ is a partition of $B$, so a bijective correspondence exists.
  }
}

\bx{
  \ea{
    \item $S'$ is an equivalence relation because we can imagine partitions of
    $y - x = z$ for $z \in \mathbb{Z}$.
    In $S$, we notice that $y - x = 1$, and $1 \in \mathbb{Z}$, so we know every
    relation in $S$ is also in $S'$, therefore $S \subset S'$.
    See \ref{chap1:sec3:p5:fig:1} for how to visualize these equivalence
    classes.
    \begin{figure}[H]
      \centering
      \def\domainSize{2}
      \begin{tikzpicture}
        \begin{axis}[
          axis y line = middle,
          axis x line = middle,
        ]

        \addplot[
          color=green,
          ultra thick,
          domain=-\domainSize:\domainSize,
          samples=100
        ]{
          x - 0.2
        };

        \addplot[
          color=blue,
          ultra thick,
          domain=-\domainSize:\domainSize,
          samples=100
        ]{
          x - 0.1
        };

        \addplot[
          color=red,
          ultra thick,
          domain=-\domainSize:\domainSize,
          samples=100
        ]{
          x
        };

        \addplot[
          color=blue,
          ultra thick,
          domain=-\domainSize:\domainSize,
          samples=100
        ]{
          x + 0.1
        };

        \addplot[
          color=green,
          ultra thick,
          domain=-\domainSize:\domainSize,
          samples=100
        ]{
          x + 0.2
        };

        \addplot[
          color=black,
          ultra thick,
          domain=-\domainSize:\domainSize,
          samples=100
        ]{
          x + 0.3
        };

        \end{axis}
      \end{tikzpicture}
      \caption{Plotting the partition defined by the equivalence relation}
      \label{chap1:sec3:p5:fig:1}
    \end{figure}

    \item If their intersection is empty, then this is trivially true.

    Otherwise, if we have some nonempty intersection, since we know any elements
    in this intersection are also part of some equivalence relation, all the
    equivalence relation properties apply, so this intersection is also an
    equivalence relation on $A$.

    \item \TODO: I'm pretty confused about this question. Wouldn't the intersection of all equivalence relations that contain $S$ just end up with $S$?
  }
}

\bx{
  Showing it is an order relation
  \begin{itemize}
    \item Comparability: Per the rule, we will always have $xCy$ or $yCx$.
    \item Nonreflexitivity: We have a tiebreaker rule that prevents $xCx$.
    \item Transitivity: Intuitively, there is an ordering at the highest level
    with $y-x^2$ value. If there are ties there, we check with $x$ values for
    ordering.
  \end{itemize}

  Yeah sorry I didn't feel like doing the order relation formally, but hopefully
  the geometric solution can help explain more. There's just a lot of casework
  and mundane proof so I didn't feel like writing it out.

  Geometrically, this is like the partition in \ref{chap1:sec3:p1:fig:1}, except
  parabolas that are higher up are ordered ``higher''. Within a parabola, the
  values on the left are less than the values on the right.
}

\bx{
  A restriction is a subset of the larger set, so all the order relation properties will still hold.
}

\bx{
  I assume the author wants us to check the $x^2 < y^2$, if $x^2 = y^2$, then $x < y$.

  \begin{enumerate}
    \item Comparability: If $x^2 \neq y^2$, either $x^2 < y^2$ or $x^2 > y^2$.
    If $x^2 = y^2$, then it must be the case that $x, y \neq 0$, and one of $x,
    y$ is negative and the other positive, so we either have $x > y$ or $x < y$.

    \item Nonreflexitivity: $xCx$ means $x^2 = x^2$, so we would check $x < \text{ or } > x$, but this is not possible since $x = x$.
    \item Transitivity: If $xCy$ then $x^2 < y^2$ or $x < y$, and if we have
    $yCz$ then $y^2 < z^2$ or $y < z$. In all cases, we can conclude $x^2 < z^2$
    or $x < z$. I'm being lazy with casework here.
  \end{enumerate}
}

\bx{
  We want to check that a dictionary order relation is an order relation.

  \begin{enumerate}
    \item Comparability: For any two $(a_1, b_1), (a_2, b_2)$, we have 2 cases:
      \begin{enumerate}
        \item $a_1 <_A a_2$ or $a_2 <_A a_1$ since $<_A$ is an order relation on $A$.
        Then we know $(a_1, b_1) < (a_2, b_2)$ or $(a_1, b_1) > (a_2, b_2)$.
        \item $a_1 = a_2$. Then we use the same argument with $b_1, b_2$, that
        either $b_1 <_B b_2$ or $b_2 <_B b_1$, which then shows the
        corresponding $<$ and $>$ on the tuple.
      \end{enumerate}

    \item Nonreflexitivity: if we have some $(a, b)$, we know by $<_A$ that $a
    <_A a$ does not hold, so the overall order relation is not possible.

    \item Transitivity: If we have $(a_1, b_1) < (a_2, b_2) < (a_3, b_3)$, then we have 2 cases for the first tuple, and 2 cases for the second tuple.
    \begin{enumerate}
      \item $a_1 <_A a_2 <_A a_3$: then we can use the transitive property of $<_A$
      \item $a_1 <_A a_2 = a_3$: we can see that $a_1 <_A a_3$
      \item $a_1 = a_2 <_A a_3$:  we can see that $a_1 <_A a_3$
      \item $a_1 = a_2 = a_3$: then we must have $b_1 <_B b_2 <_B b_3$, so we can use the transitive property of $<_B$
    \end{enumerate}
  \end{enumerate}
}

\bx{
  \begin{itemize}
    \item One way to see that this is an order preserving function is that the derivative is always positive between $(-1, 1)$,
    \begin{equation*}
      f'(x) = \frac{
        x^2 + 1
      }{
        \pa{1-x^2}^2
      },
    \end{equation*}
    which means the function is monotonically increasing, and thus will preserve the order, since monotonically increasing functions have the property that
    \begin{equation*}
      a < b \implies f(a) < f(b).
    \end{equation*}

    \item This is just an algebra exercise...pretty easy to verify
  \end{itemize}
}

\bx{
  \AFSOC there is more than one immediate successor to some $a$, call them $b$ and $c$. Then by order set properties, we know that either $b < c$ or $c < b$.
  In either case, we end up finding that $b$ or $c$ cannot be immediate successors, since for example, if $b < c$, we have that $(a, c)$ is not empty.

  The argument for immediate predecessor is symmetric to this argument.

  To show there can only be one smallest element, we can \AFSOC there is more
  than one. If we call these $a, a'$, we know from ordering properties that WLOG
  $a < a'$, then $a'$ is not the smallest element, so this is a contradiction.

  The argument for the largest element is symmetric.
}

\bx{
  \begin{enumerate}[label=(\roman**)]
    \item Every element has an immediate predecessor. For some $(x, y)$, the
    immediate predecessor is $(x, y+1)$.
    There is no smallest element, since you can always find a smaller element,
    i.e. for any $(x, y)$, $(x-1, y) < (x, y)$.

    \item The immediate predecessor for some $(x, y)$ is $(x+1, y+1)$. There is
    no smallest element, since you can always find $(x-1, y) < (x, y)$ for any
    $(x, y)$.

    \item The immediate predecessor for some $(x, y)$ is $(x-1, y+1)$. There is
    no smallest element, since for any $(x, y)$, you have $(x-1, y) < (x, y)$.
  \end{enumerate}

  Not rigorous, but geometrically, the first ordering is like a zigzag on the
  plane, the second one is like $y = x + C$, and the third is $y = -x + C$, so
  these orderings are all different.
}

\bx{
  Suppose $A$ has the least upper bound property, meaning every nonempty subset
  $A_0$ of $A$ is bounded above by some least upper bound.
  \AFSOC $A$ does not have the greatest lower bound property, that is $\exists
  A_1 \subset A$ such that $A_1$ does not have a greatest lower bound.
  If this is the case, then consider the lower bound $a$ for this set $A_1$, and
  consider the set $A_2$, which we define as
  \begin{equation*}
    A_2 = \pbrac{a' \mid a' \geq a}
  \end{equation*}
  Notice that $A_2$ cannot be empty, or else $a$ is the only, and therefore greatest lower bound for $A_1$.
  Now, from the definition of $A_2$, we can see that $a$ is an upper bound for
  $A_2$. However, we now claim that there is no least upper bound for $A_2$.
  Because if there were, call it some $a_2$, then $a_2$ would be the greatest
  lower bound for $A_1$, since $a_2$ is larger than all lower bounds of $A_1$.
  This is a contradiction, since we assumed that $A_1$ does not have a greatest lower bond.
  Therefore, we must conclude that every subset of $A$ has a greatest lower bound.
}

\bx{
  \ea{
    \item If $C$ is symmetric, then $(a, b) \in C \implies (b, a) \in C$, which means $D \subset C$.

    If $C = D$, then $(a, b) \in C \implies (b, a) \in D = C$, so therefore $C$ is symmetric.

    \item We will check the order relation properties for $D$
    \begin{enumerate}
      \item Comparability: For any $(b, a) \in D$, we know $(a, b) \in C$, so we
      know either $b < a$ or $a < b$.
      \item Nonreflexitivity: $(b, b) \in D$ would imply $(b, b) \in C$, but $C$
      is an order relation so this is not possible.
      \item Transitivity: $(c, b) \in D, (b, a) \in D$. We know that $(a, b),
      (b, c) \in C$, so we know $(a, c) \in C$, so therefore $(c, a) \in D$,
      which proves transitivity.
    \end{enumerate}

    \item The other direction of the argument is symmetric.
  }
}

\bx{
  \ea{
    \item Let us show that
    \begin{itemize}
      \item 1 is the least upper bound for $[0, 1]$. We can see this because if
      you pick any smaller of an upper bound $u$, $u < 1$ and thus is not an
      upper bound for the set.

      \item 1 is the least upper bound for $[0, 1)$. Suppose we have some other
      least upper bound $u$ such that $x \in [0, 1), x \leq u$, but $u < 1$.
      Then consider $u' = u + \epsilon/2$, where $\epsilon = \frac{1-u}{2}$.
      Then $u'$ is $<1$ but $u' > u$ and is also an upper bound, which means $u$
      was not the least upper bound. Therefore we have reached a contradiction
      and conclude that 1 is the least upper bound.
    \end{itemize}

    \item $[0, 1] \times [0, 1]$ with dictionary ordering has least upper bound property.
    This is because for any subset, if we look at the first coordinate, it is in
    $[0, 1]$, which we showed has the least upper bound property, so call this
    upper bound $u_1$.
    Similarly, for the second coordinate it is also in $[0, 1]$, so we have a
    least upper bound $u_2$ for this coordinate.
    Then we have $(u_1, u_2)$ is a least upper bound for any subset in $[0, 1] \times [0, 1]$.

    This argument holds for $[0, 1] \times [0, 1)$ and $[0, 1) \times [0, 1]$.
  }
}